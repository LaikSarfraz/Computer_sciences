\documentclass{article}

\usepackage{url}
\usepackage[T1]{fontenc}        % pour « »
\usepackage[utf8]{inputenc}     % entrée en utf-8
\usepackage[dvips,dvipdf]{graphicx}
\usepackage{color}
\usepackage[french]{babel}
\usepackage{a4}

\setlength{\parindent}{0cm}

\newcommand{\xxx}{\vspace*}

\newcounter{questionnumero}
\stepcounter{questionnumero}
\newcommand{\Question}{{\bf Question \arabic{questionnumero}~:~}\stepcounter{questionnumero}}

\title{Cours de compilation\\ TD numéro 2}
\date{8 octobre 2015}
\author{Jean Méhat}

\begin{document}

\maketitle

\Question Écrire en assembleur une fnoction {\tt min2} qui reçoit deux
arguments et renvoie le plus petit des deux.

\Question
Traduire en assembleur la fonction {\tt indexa} suivante, qui renvoie
l'adresse du premier a dans une chaîne (ou 0 s'il n'y est pas).
\begin{verbatim}
char *
indexa(char string[]){
        int i;
        
        for(i = 0; string[i] != 0; i++)
                if (string[i] == 'a')
                        return & string[i];
        return 0;
}
\end{verbatim}
ou bien :
\begin{verbatim}
char * indexa(char * p){
        for(; *p; p++)
                if (*p == 'a')
                        return p
        return 0;
}
\end{verbatim}

\Question
Écrire une fonction {\tt rindexa} qui renvoie l'adresse du
{\em dernier} caractère 'a' dans la chaîne.

\Question
Traduire la fonction suivante du C vers l'assembleur
\begin{verbatim}
int
fact(int n){
        int r;

        for(r = 1; n > 1; n--)
                r *= n;
        return r;
}
\end{verbatim}

\Question
Traduire la fonction suivante du C vers l'assembleur
\begin{verbatim}
int
fib(int n){
        if (n < 2)
                return n;
        return fib(n - 1) + fib(n - 2);
}
\end{verbatim}


\Question
Traduire l'assembleur suivant en C
\begin{verbatim}
        .text
        .globl  heron
heron:
        pushl   %ebx
        movl    8(%esp),%eax
        movl    12(%esp),%ebx
        movl    16(%esp),%ecx
        movl    %eax,%edx
        addl    %ebx,%eax
        addl    %ecx,%eax
        sarl    $1,%eax
        subl    %eax,%ebx
        subl    %eax,%ecx
        subl    %eax,%edx
        imull   %ebx,%eax
        imull   %ecx,%eax
        imull   %edx,%eax
        negl    %eax
        popl    %ebx
        ret
\end{verbatim}

\section*{Bonus}


\Question Comparer la vitesse d'exécution de votre fonction assembleur
          {\tt indexa} avec celle de la fonction {\tt strchr} de la
          bibliothèque. (À étudier selon la longueur de la chaîne et la
          position du caractère recherché ; on peut aussi comparer
          avec la vitesse de la fonction en C du sujet compilée avec et
          sans optimisations.)

\Question Écrire en assembleur une fonction {\tt médian3} qui reçoit trois
arguments et renvoie celui qui n'est ni le plus grand, ni le plus petit.
La fonction est utile dans Quick Sort pour choisir le pivot entre le
premier élément, le dernier et celui du milieu.

\end{document}
