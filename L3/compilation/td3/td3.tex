\documentclass{article}

\usepackage{url}
\usepackage[T1]{fontenc}        % pour « »
\usepackage[utf8]{inputenc}     % entrée en utf-8
\usepackage[dvips,dvipdf]{graphicx}
\usepackage{color}
\usepackage[french]{babel}
\usepackage{a4}

\setlength{\parindent}{0cm}

\newcommand{\xxx}{\vspace*}

\newcounter{questionnumero}
\stepcounter{questionnumero}
\newcommand{\Question}{{\bf Question \arabic{questionnumero}~:~}\stepcounter{questionnumero}}

\title{ TD numéro 3}
\date{25 octobre 2015}
\author{Sarfraz Laïk}

\begin{document}
\maketitle
Réponse au TD

\begin{itemize}
\item[1] 1/Pushq remplace pushl. De plus c'est le registre rbp qui pointe sur le sommet de la pile.
\item[2] 2/Les valeurs sont toujours renvoyées avec le registre eax
\item[3] 3/ Une fonction renvoie un pointeur en mettant la variable globale dans le registre eax
\item[4] 4/Si une fonction n'a pas d'arguments, après un call, il ya l'instruction nop qui suit.
\item[5] 5/ les 6 premiers arguments vont dans les registres edi , esi , edx , ecx , r8d puis r9d . ensuite dans la pile avec pushq. Cette réponse était dans le pdf.
\item[6] 6/ Pour 6 boucles, ce sont les registres de r15 à rbx qui sont disponibles pour stocker des variables entieres. Il y a 6 registres disponibles.
\item[7] 7/ Le fichier se trouve dans le dossier suivant /usr/lib/gcc/x86_64-linux-gnu/4.9/include/stdarg.h . 
\end{itemize}


\end{document}
